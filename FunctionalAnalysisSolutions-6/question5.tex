\section{Question 5}

\horz

Let $H$ be a Hilbert space and $P_1, P_2\in \mathcal B(H),$ are two orthogonal projections in $\mathcal B(H).$  Show that $P_1 +P_2$ is an orthogonal projection if and only if $ran(P_1)$ is orthogonal to $ran(P_2),$ that $\langle P_1x, P_2y\rangle =0$ for every $x,y\in H.$ In this case $ran (P_1+P_2)= ran(P_1) + ran (P_2).$

\horz

\begin{proof}
    Let $H$ be a Hilbert space and let $P_{1}, P_{2}$ be two orthogonal projections in $\calB \left( H \right)$. We proceed to show that $P_{1} + P_{2}$ is an orthogonal projection iff $\im P_{1}$ is orthogonal to $\im P_{2}$. 

    \begin{description}
	\item[$\left( \Longrightarrow \right)$] Suppose that $P_{1} + P_{2}$ is an orthogonal projection. Let $y_{1} \in \im P_{1}$ and $y_{2} \in \im P_{2}$. Then we have that $y_{1} = P_{1}x_{1}$ and $y_{2} =P_{2}x_{2}$ for some $x_{1} \in H$ and some $x_{2} \in H$.

	    First, we show that $P_{1} P_{2} = -P_{1} P_{2}$. Note that
	    \begin{align*}
		P_{1} + P_{2} &=  \left( P_{1} + P_{2} \right)^{2} \\
		&= P_{1} + P_{1}P_{2} + P_{2} P_{1} + P_{2} \\
	    \end{align*}
	    Hence, this implies that
	    \begin{equation}
		P_{1}P_{2}=-P_{2}P_{1}.
		\label{eqn:5.1}
	    \end{equation}
	    Now, multiplying the previous equation by $P_{2}$, we have that $P_{2}P_{1}=-P_{2}P_{1}P_{2}$. $P_{2}P_{1}$ is then self-adjoint as
	    \begin{equation*}
		(P_{2}P_{1})^{*} = \left( -P_{2}P_{1}P_{2} \right)^{*} = - P_{2}^{*} P_{1}^{*} P_{2}^{*} = -P_{2} P_{1} P_{2}=P_{2}P_{1}.
	    \end{equation*}
	    Taking adjoint in both sides of Equation \ref{eqn:5.1}, we have that 
	    \begin{align*}
		P_{2}P_{1}&=\left( P_{2}P_{1} \right)^{*} \\
		&= \left( -P_{1}P_{2} \right)^{*} \\
		&= -P_{2}P_{1}
	    \end{align*}
	    and consequently,
	    \begin{equation}
		P_{2}P_{1} = P_{1}P_{2} = 0.
		\label{eqn:5.2}
	    \end{equation}

	    Now, we proceed to show that $\im P_{1}$ is orthogonal to $\im P_{2}$. To this end, let $x_{1}, x_{2} \in H$. Then we have that
	    \begin{align*}
		\ip{P_{1}x_{1}, P_{2}x_{2}} &= \ip{P_{2}P_{1}x_{1},x_{2}}\\
		&= \ip{0x_{1},x_{2}} \\
		&= 0 & (\text{See } \ref{eqn:5.2})
	    \end{align*}
	    This shows that $\im P_{1}$ is orthogonal to $\im P_{2}$.

	\item[$\left( \Longleftarrow \right)$] Suppose that $\im P_{1}$ is orthogonal to $\im P_{2}$. We wish to show that $P_{1} + P_{2}$ is an orthogonal projection. The fact that $P_{1} + P_{2}$ is selfadjoint is immediate. It remains to show that $P_{1} + P_{2}$ is idempotent.

	    But before that, we show that $P_{1}P_{2}=0$. Let $f\in H$. Observe that
	    \begin{align*}
		\norm{P_{1}P_{2}f}^{2} &= \ip{P_{1}P_{2}f, P_{1}P_{2}f} \\
		&= \ip{P_{1}^{*}P_{1}P_{2}f, P_{2}f} \\
		&= \ip{P_{1}^{2}P_{2}f, P_{2}f} \\
		&= \ip{P_{1}P_{2}f, P_{2}f} \\
		&= 0.
	    \end{align*}
	    The first equality is by definition, the second is by definition of adjoint operator and the third is because $P_{1}$ is an orthogonal projection and hence adjoint, the the third is because $P_{1}$ is selfadjoint and the last is due to our assumption that $\im P_{1}$ is orthogonal to $\im P_{2}$. This shows that $P_{1}P_{2} = 0$.

	    Similarly, it can be shown that $P_{2}P_{1} = 0$. Now, we proceed to $(P_{1} + P_{2})^{2} = P_{1} + P_{2}$. Observe that
	    \begin{align*}
		(P_{1} + P_{2})^{2} &= P_{1}^{2} + P_{1}P_{2} + P_{2}P_{1} + P_{2}^{2} \\
		&= P_{1} + P_{2}.
	    \end{align*}
	    The first equality is obvious and in the second one, we are using the fact that $P_{1}, P_{2}$ are idempotent and $P_{1}P_{2}=P_{2}P_{1} = 0$.
    \end{description}
    Now, we proceed to show that $\im \left( P_{1} + P_{2} \right) = \im P_{1} + \im P_{2}$. Consider the following:
    \begin{description}
	\item[$\left( \subset \right)$:] Let $y \in \im \left( P_{1} + P_{2} \right)$. Then $y=\left( P_{1} + P_{2} \right)x$ for some $x\in H$. Then we have that 
	    \begin{equation*}
		y=\underbrace{P_{1}x}_{\in \im P_{1}} + \underbrace{P_{2}x}_{\in \im P_{2}} \in \im P_{1} + \im P_{2}.
	    \end{equation*}
	\item[$\left( \supset \right)$:] Let $y\in \im P_{1} + \im P_{2}$. Then $y= P_{1}x_1 + P_{2}x_{2}$ for some $x_{1} \in H$ and some $x_{2} \in H$. Since we have that $P_{1}+P_{2}$ is an orthogonal projection, we have that $P_{1}P_{2}=-P_{2}P_{1}$ as in $\left( \Longrightarrow \right)$ in the first part of the proof. Thus, we have that
	    \begin{equation*}
		P_{2}y= P_{2}P_{1}x_{1} + P_{2}^{2}x_{2} = P_{2}x_{2}
	    \end{equation*}
	    and similarly we have that
	    \begin{equation*}
		P_{1}y=P_{1}x_{1}.
	    \end{equation*}
	    Hence we have that
	    \begin{equation*}
		y=P_{1}y + P_{2}y = \left( P_{1} + P_{2} \right)y
	    \end{equation*}
	    and consequently $y \in \im \left( P_{1} + P_{2} \right)$.
    \end{description}
    This completes the proof.
\end{proof}
