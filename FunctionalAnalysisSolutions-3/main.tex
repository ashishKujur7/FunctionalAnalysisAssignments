\documentclass[12pt]{article}
\usepackage[margin=1in]{geometry}
\usepackage{amsfonts, amsmath}
\usepackage[T1]{fontenc}
\usepackage{mathrsfs, enumitem}
\usepackage{hyperref}
\usepackage[utf8]{inputenc}
\usepackage{amssymb}
\usepackage{amsfonts}
\usepackage{amsmath}
\usepackage{amsthm}
\usepackage{color}
\usepackage{hyperref}
\usepackage{csquotes}
%\usepackage{fourier}
\usepackage{tikz-cd}
\usepackage{lipsum}
\usepackage{cancel, eucal}

\newtheorem{theorem}{Theorem}[subsection]
\newtheorem{lemma}[theorem]{Lemma}
\newtheorem{claim}[theorem]{Claim}
\newtheorem{proposition}[theorem]{Proposition}
\newtheorem{corollary}[theorem]{Corollary}
\newtheorem{fact}[theorem]{Fact}
\newtheorem{notation}[theorem]{Notation}
\newtheorem{observation}[theorem]{Observation}
\newtheorem{conjecture}[theorem]{Conjecture}
\newtheorem{exercise}[theorem]{Exercise}

\theoremstyle{definition}
\newtheorem{definition}[theorem]{Definition}
\newtheorem{example}[theorem]{Example}
\numberwithin{equation}{subsection}

\theoremstyle{remark}
\newtheorem{remark}[theorem]{Remark}
\theoremstyle{plain}
\newcommand{\ignore}[1]{}

% section symbol
%\renewcommand{\thesection}{\S\arabic{section}}

% \renewcommand{\Pr}{{\bf Pr}}
% \newcommand{\Prx}{\mathop{\bf Pr\/}}
% \newcommand{\E}{{\bf E}}
% \newcommand{\Ex}{\mathop{\bf E\/}}
% \newcommand{\Var}{{\bf Var}}
% \newcommand{\Varx}{\mathop{\bf Var\/}}
% \newcommand{\Cov}{{\bf Cov}}
% \newcommand{\Covx}{\mathop{\bf Cov\/}}

% shortcuts for symbol names that are too long to type
\newcommand{\eps}{\epsilon}
\newcommand{\lam}{\lambda}
\renewcommand{\l}{\ell}
\newcommand{\la}{\langle}
\newcommand{\ra}{\rangle}
\newcommand{\wh}{\widehat}
\newcommand{\wt}{\widetilde}

% % "blackboard-fonted" letters for the reals, naturals etc.
\newcommand{\R}{\mathbb R}
\newcommand{\N}{\mathbb N}
\newcommand{\Z}{\mathbb Z}
\newcommand{\F}{\mathbb F}
\newcommand{\Q}{\mathbb Q}
\newcommand{\C}{\mathbb C}

% % operators that should be typeset in Roman font
% \newcommand{\poly}{\mathrm{poly}}
% \newcommand{\polylog}{\mathrm{polylog}}
% \newcommand{\sgn}{\mathrm{sgn}}
% \newcommand{\avg}{\mathop{\mathrm{avg}}}
% \newcommand{\val}{{\mathrm{val}}}

% % complexity classes
% \renewcommand{\P}{\mathrm{P}}
% \newcommand{\NP}{\mathrm{NP}}
% \newcommand{\BPP}{\mathrm{BPP}}
% \newcommand{\DTIME}{\mathrm{DTIME}}
% \newcommand{\ZPTIME}{\mathrm{ZPTIME}}
% \newcommand{\BPTIME}{\mathrm{BPTIME}}
% \newcommand{\NTIME}{\mathrm{NTIME}}

% values associated to optimization algorithm instances
\newcommand{\Opt}{{\mathsf{Opt}}}
\newcommand{\Alg}{{\mathsf{Alg}}}
\newcommand{\Lp}{{\mathsf{Lp}}}
\newcommand{\Sdp}{{\mathsf{Sdp}}}
\newcommand{\Exp}{{\mathsf{Exp}}}

% if you think the sum and product signs are too big in your math mode; x convention
% as in the probability operators
\newcommand{\littlesum}{{\textstyle \sum}}
\newcommand{\littlesumx}{\mathop{{\textstyle \sum}}}
\newcommand{\littleprod}{{\textstyle \prod}}
\newcommand{\littleprodx}{\mathop{{\textstyle \prod}}}

% horizontal line across the page
\newcommand{\horz}{
\vspace{-.4in}
\begin{center}
\begin{tabular}{p{\textwidth}}\\
\hline
\end{tabular}
\end{center}
}

% calligraphic letters
\newcommand{\calA}{{\cal A}}
\newcommand{\calB}{{\cal B}}
\newcommand{\calC}{{\cal C}}
\newcommand{\calD}{{\cal D}}
\newcommand{\calE}{{\cal E}}
\newcommand{\calF}{{\cal F}}
\newcommand{\calG}{{\cal G}}
\newcommand{\calH}{{\cal H}}
\newcommand{\calI}{{\cal I}}
\newcommand{\calJ}{{\cal J}}
\newcommand{\calK}{{\cal K}}
\newcommand{\calL}{{\cal L}}
\newcommand{\calM}{{\cal M}}
\newcommand{\calN}{{\cal N}}
\newcommand{\calO}{{\cal O}}
\newcommand{\calP}{{\cal P}}
\newcommand{\calQ}{{\cal Q}}
\newcommand{\calR}{{\cal R}}
\newcommand{\calS}{{\cal S}}
\newcommand{\calT}{{\cal T}}
\newcommand{\calU}{{\cal U}}
\newcommand{\calV}{{\cal V}}
\newcommand{\calW}{{\cal W}}
\newcommand{\calX}{{\cal X}}
\newcommand{\calY}{{\cal Y}}
\newcommand{\calZ}{{\cal Z}}

% bold letters (useful for random variables)
%----------------------------------------------
% \renewcommand{\a}{{\boldsymbol a}}
% \renewcommand{\b}{{\boldsymbol b}}
% \renewcommand{\c}{{\boldsymbol c}}
% \renewcommand{\d}{{\boldsymbol d}}
% \newcommand{\e}{{\boldsymbol e}}
% \newcommand{\f}{{\boldsymbol f}}
% \newcommand{\g}{{\boldsymbol g}}
% \newcommand{\h}{{\boldsymbol h}}
% \renewcommand{\i}{{\boldsymbol i}}
% \renewcommand{\j}{{\boldsymbol j}}
% \renewcommand{\k}{{\boldsymbol k}}
% \newcommand{\m}{{\boldsymbol m}}
% \newcommand{\n}{{\boldsymbol n}}
% \renewcommand{\o}{{\boldsymbol o}}
% \newcommand{\p}{{\boldsymbol p}}
% \newcommand{\q}{{\boldsymbol q}}
% \renewcommand{\r}{{\boldsymbol r}}
% \newcommand{\s}{{\boldsymbol s}}
% \renewcommand{\t}{{\boldsymbol t}}
% \renewcommand{\u}{{\boldsymbol u}}
% \renewcommand{\v}{{\boldsymbol v}}
% \newcommand{\w}{{\boldsymbol w}}
% \newcommand{\x}{{\boldsymbol x}}
% \newcommand{\y}{{\boldsymbol y}}
% \newcommand{\z}{{\boldsymbol z}}
% \newcommand{\A}{{\boldsymbol A}}
% \newcommand{\B}{{\boldsymbol B}}
% \newcommand{\C}{{\boldsymbol C}}
% \newcommand{\D}{{\boldsymbol D}}
% \newcommand{\E}{{\boldsymbol E}}
% \newcommand{\F}{{\boldsymbol F}}
% \newcommand{\G}{{\boldsymbol G}}
% \renewcommand{\H}{{\boldsymbol H}}
% \newcommand{\I}{{\boldsymbol I}}
% \newcommand{\J}{{\boldsymbol J}}
% \newcommand{\K}{{\boldsymbol K}}
% \renewcommand{\L}{{\boldsymbol L}}
% \newcommand{\M}{{\boldsymbol M}}
% \renewcommand{\O}{{\boldsymbol O}}
% \renewcommand{\P}{{\mathbb{P}}}
% \newcommand{\Q}{{\boldsymbol Q}}
% \newcommand{\R}{{\boldsymbol R}}
% \renewcommand{\S}{{\boldsymbol S}}
% \newcommand{\T}{{\boldsymbol T}}
% \newcommand{\U}{{\boldsymbol U}}
% \newcommand{\V}{{\boldsymbol V}}
% \newcommand{\W}{{\boldsymbol W}}
% \newcommand{\X}{{\boldsymbol X}}
% \newcommand{\Y}{{\boldsymbol Y}}
% \newcommand{\Z}{{\boldsymbol Z}}

% script letters
\newcommand{\scrA}{{\mathscr A}}
\newcommand{\scrB}{{\mathscr B}}
\newcommand{\scrC}{{\mathscr C}}
\newcommand{\scrD}{{\mathscr D}}
\newcommand{\scrE}{{\mathscr E}}
\newcommand{\scrF}{{\mathscr F}}
\newcommand{\scrG}{{\mathscr G}}
\newcommand{\scrH}{{\mathscr H}}
\newcommand{\scrI}{{\mathscr I}}
\newcommand{\scrJ}{{\mathscr J}}
\newcommand{\scrK}{{\mathscr K}}
\newcommand{\scrL}{{\mathscr L}}
\newcommand{\scrM}{{\mathscr M}}
\newcommand{\scrN}{{\mathscr N}}
\newcommand{\scrO}{{\mathscr O}}
\newcommand{\scrP}{{\mathscr P}}
\newcommand{\scrQ}{{\mathscr Q}}
\newcommand{\scrR}{{\mathscr R}}
\newcommand{\scrS}{{\mathscr S}}
\newcommand{\scrT}{{\mathscr T}}
\newcommand{\scrU}{{\mathscr U}}
\newcommand{\scrV}{{\mathscr V}}
\newcommand{\scrW}{{\mathscr W}}
\newcommand{\scrX}{{\mathscr X}}
\newcommand{\scrY}{{\mathscr Y}}
\newcommand{\scrZ}{{\mathscr Z}}

\newcommand{\im}{{\text{im }}}
\newcommand{\ip}[1]{\left\langle #1 \right\rangle}
\newcommand{\norm}[1]{\left\lVert #1 \right\rVert}
\newcommand{\abs}[1]{\left\lvert #1 \right\rvert}

\newcommand\blfootnote[1]{%
  \begingroup
  \renewcommand\thefootnote{}\footnote{#1}%
  \addtocounter{footnote}{-1}%
  \endgroup
}

\title{Functional Analysis Assignment 3}
\author{\textsc{Ashish Kujur}}

\date{}

\begin{document}

\maketitle
\section*{Note}
A checkmark $\checkmark$ indicates the question has been done.
\tableofcontents
\section{Question 1}
\horz
Let $V$ and $W$ be two NLS and $T: V \rightarrow W$ be a linear map. Show that $T$ is continuous if and only if $T$ maps every Cauchy sequence of $V$ to a Cauchy sequence of $W$.
\horz

\begin{proof}
    Let $V,W$ be two NLS and let $T: V\to W$ be a linear map.

    $\left( \Longrightarrow \right)$ Suppose that $T$ is continuous. Let $\left\{ x_{n} \right\}$ be a Cauchy sequence in $X$. We want to show that $\left\{ Tx_{n} \right\}$ is Cauchy sequence in $Y$. To do so, let $\varepsilon > 0$ be given. By the continuity of $T$, there is some $k> 0$ such that
    \begin{equation}
	\norm{Tx}\le k\norm{x} \text{ for every } x \in X\text{.}
	\label{eqn:q1-continuity-of-T}
    \end{equation}
    Since $\left\{ x_{n} \right\}$ is Cauchy, there is some $N\in \N$ such that
    \begin{equation}
	\norm{x_{n}-x_{m}} < \frac{\varepsilon}{k} \text{ for every } n,m \ge N
	\label{eqn:q1-cauchy}
    \end{equation}
    Thus, for every $n,m \ge N$, we have that
    \begin{align*}
	\norm{Tx_{n} - Tx_{m}} & \le k \norm{x_{n} - x_{m}} & \text{from } \ref{eqn:q1-continuity-of-T} \\
	& < \varepsilon & \text{from } \ref{eqn:q1-cauchy}
    \end{align*}
    This shows that $\left\{ Tx_{n} \right\}$ is Cauchy in $Y$.

    $\left( \Longleftarrow \right)$ We prove it by contrapostitively. Suppose that $T$ is not continuous. Then for every $k > 0$,
    \begin{align*}
	\norm{Tx} &> k \norm{x} \text{ for some } x \in X\text{.}
    \end{align*}
    Thus, for each $n\in \N$, we can find some $x_{n} \in X$ such that $\norm{Tx_{n}} > n^{2} \norm{x_{n}}$. Consider the sequence $\left\{ y_{n} \right\}$ in $V$ defined by 
    \begin{align*}
	y_{n} = \frac{x_{n}}{n\norm{x_{n}}} \text{ for each } n \in \N
    \end{align*}
    We now show that $\left\{ y_{n} \right\}$ is Cauchy. Let $\varepsilon > 0$ be given. Select $N \in \N$ such that $\frac{2}{N}<\varepsilon$. For $k \in \N$ and $n\ge N$, we have that
    \begin{align*}
    \norm{y_{n+k} - y_{m}} &= \norm{\frac{x_{n+k}}{(n+k)\norm{x_{n+k}}}-  \frac{x_{n}}{n\norm{x_{n}}}} \\
    &\le \frac{1}{n+k} + \frac{1}{n} \\
    &= \frac{2}{n} \le \frac{2}{N} < \varepsilon
    \end{align*}
    This shows that $\left\{ y_{n} \right\}$ is Cauchy but on the other hand, we have that
    \begin{align*}
	\norm{Ty_{n}} &= \norm{T\left( \frac{x_{n}}{n\norm{x_{n}}} \right)} >n
    \end{align*}
    This shows that $\left\{ Ty_{n} \right\}$ is unbounded, a property which Cauchy sequences cannot have.
\end{proof}

\section{Question 2}
\horz
Let $X$ and $Y$ be two Banach spaces and  $T: X \rightarrow Y$ be a continuous linear transformation. Show that there exist a constant $c>0$ such that $\|Tx\| \geqslant c \|x\|$ for all $x\in X$ if and only if $\ker T=\{0\}$ and $\im (T)$ is closed.

\horz

\begin{proof}[Solution]
    $\left( \Longrightarrow \right)$ Suppose that there is a constant $c>0$ such that $\norm{Tx} \ge c\norm{x}$ for all $x\in X$. 

    First, let us show that $\ker T = \left\{ 0 \right\}$. Let $x\in \ker T$. Then $Tx=0$. Then we have that $0=\norm{Tx}\ge c \norm{x}$ and hence $x=0$. 

    To show that that the image of $T$ is closed, let $\left( Tx_{n} \right)_{n\in \N}$ be a sequence converging to some $y \in Y$. We need to show that $y=Tx$ for some $x\in X$. 

    Since $\left( Tx_{n} \right)$ is convergent, it is Cauchy in $Y$. Therefore, we have that 
    \begin{align*}
	\norm{x_{n} -x_m} &\le \frac{1}{c} \norm{Tx_{n} - Tx_{m}}
    \end{align*}
    for all $m,n \in \N$. This shows that $\left( x_{n} \right)_{n\in \N}$ is Cauchy in $X$. Since $X$ is Banach, we have that $\left( x_{n} \right)_{n\in\N}$ converges to some $x\in X$. By continuity, we have that $\left( Tx_{n} \right)_{n\in \N}$ converges to $Tx$. By uniqueness of limits, we have that $Tx=y$.

    $\left( \Longleftarrow \right)$ If $X=\left\{ 0 \right\}$ then the result is trivial. Suppose that $X \ne \left\{ 0 \right\}$. Since $T: X \to Y$ is injective, we consider the map $T^{-1} : \im T \to X$. Note that $T$ is bounded linear transformation, thus, $T^{-1}$ is a bounded linear transformation by the inverse mapping theorem. (Quick remark: $\im T$ is Banach by virtue of being closed).

    Thus, we have that
    \begin{align*}
	\norm{x} &= \norm{T^{-1} \left( Tx \right)} \\
	&\le \norm{T^{-1}} \norm{Tx}
    \end{align*}
    for any $x\in X$.
    We will be done if we show that $\norm{T^{-1}} \ne 0$. Since $X$ is nonzero, $\im T$ is nonzero. Select a nonzero vector $y\in \im (T)$ such that $\norm{y} \le 1$. Thus, we have that $\norm{T^{-1}} \ge \norm{T^{-1} \left( y \right)}$. Hence $y=Tx$ for some nonzero $x\in X$. Thus, $\norm{T^{-1}} \ge \norm{x} >0$. This completes the proof.
\end{proof}

\end{document}
