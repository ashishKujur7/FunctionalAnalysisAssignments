\documentclass[12pt]{article}
\usepackage[margin=1in]{geometry}
\usepackage{amsfonts, amsmath}
\usepackage[T1]{fontenc}
\usepackage{mathrsfs, enumitem}
\usepackage{hyperref}
\usepackage[utf8]{inputenc}
\usepackage{amssymb}
\usepackage{amsfonts}
\usepackage{amsmath}
\usepackage{amsthm}
\usepackage{color}
\usepackage{hyperref}
\usepackage{csquotes}
%\usepackage{fourier}
\usepackage{tikz-cd}
\usepackage{lipsum}
\usepackage{cancel, eucal}
\usepackage{wasysym}

\newtheorem{theorem}{Theorem}[subsection]
\newtheorem{lemma}[theorem]{Lemma}
\newtheorem{claim}[theorem]{Claim}
\newtheorem{proposition}[theorem]{Proposition}
\newtheorem{corollary}[theorem]{Corollary}
\newtheorem{fact}[theorem]{Fact}
\newtheorem{notation}[theorem]{Notation}
\newtheorem{observation}[theorem]{Observation}
\newtheorem{conjecture}[theorem]{Conjecture}
\newtheorem{exercise}[theorem]{Exercise}

\theoremstyle{definition}
\newtheorem{definition}[theorem]{Definition}
\newtheorem{example}[theorem]{Example}
\numberwithin{equation}{subsection}

\theoremstyle{remark}
\newtheorem{remark}[theorem]{Remark}
\theoremstyle{plain}
\newcommand{\ignore}[1]{}

% section symbol
%\renewcommand{\thesection}{\S\arabic{section}}

% \renewcommand{\Pr}{{\bf Pr}}
% \newcommand{\Prx}{\mathop{\bf Pr\/}}
% \newcommand{\E}{{\bf E}}
% \newcommand{\Ex}{\mathop{\bf E\/}}
% \newcommand{\Var}{{\bf Var}}
% \newcommand{\Varx}{\mathop{\bf Var\/}}
% \newcommand{\Cov}{{\bf Cov}}
% \newcommand{\Covx}{\mathop{\bf Cov\/}}

% shortcuts for symbol names that are too long to type
\newcommand{\eps}{\epsilon}
\newcommand{\lam}{\lambda}
\renewcommand{\l}{\ell}
\newcommand{\la}{\langle}
\newcommand{\ra}{\rangle}
\newcommand{\wh}{\widehat}
\newcommand{\wt}{\widetilde}

% % "blackboard-fonted" letters for the reals, naturals etc.
\newcommand{\R}{\mathbb R}
\newcommand{\N}{\mathbb N}
\newcommand{\Z}{\mathbb Z}
\newcommand{\F}{\mathbb F}
\newcommand{\Q}{\mathbb Q}
\newcommand{\C}{\mathbb C}

% % operators that should be typeset in Roman font
% \newcommand{\poly}{\mathrm{poly}}
% \newcommand{\polylog}{\mathrm{polylog}}
% \newcommand{\sgn}{\mathrm{sgn}}
% \newcommand{\avg}{\mathop{\mathrm{avg}}}
% \newcommand{\val}{{\mathrm{val}}}

% % complexity classes
% \renewcommand{\P}{\mathrm{P}}
% \newcommand{\NP}{\mathrm{NP}}
% \newcommand{\BPP}{\mathrm{BPP}}
% \newcommand{\DTIME}{\mathrm{DTIME}}
% \newcommand{\ZPTIME}{\mathrm{ZPTIME}}
% \newcommand{\BPTIME}{\mathrm{BPTIME}}
% \newcommand{\NTIME}{\mathrm{NTIME}}

% values associated to optimization algorithm instances
\newcommand{\Opt}{{\mathsf{Opt}}}
\newcommand{\Alg}{{\mathsf{Alg}}}
\newcommand{\Lp}{{\mathsf{Lp}}}
\newcommand{\Sdp}{{\mathsf{Sdp}}}
\newcommand{\Exp}{{\mathsf{Exp}}}

% if you think the sum and product signs are too big in your math mode; x convention
% as in the probability operators
\newcommand{\littlesum}{{\textstyle \sum}}
\newcommand{\littlesumx}{\mathop{{\textstyle \sum}}}
\newcommand{\littleprod}{{\textstyle \prod}}
\newcommand{\littleprodx}{\mathop{{\textstyle \prod}}}

%for end of proof

\renewcommand\qedsymbol{\blacksmiley{}}

% horizontal line across the page
\newcommand{\horz}{
\vspace{-.4in}
\begin{center}
\begin{tabular}{p{\textwidth}}\\
\hline
\end{tabular}
\end{center}
}

% calligraphic letters
\newcommand{\calA}{{\cal A}}
\newcommand{\calB}{{\cal B}}
\newcommand{\calC}{{\cal C}}
\newcommand{\calD}{{\cal D}}
\newcommand{\calE}{{\cal E}}
\newcommand{\calF}{{\cal F}}
\newcommand{\calG}{{\cal G}}
\newcommand{\calH}{{\cal H}}
\newcommand{\calI}{{\cal I}}
\newcommand{\calJ}{{\cal J}}
\newcommand{\calK}{{\cal K}}
\newcommand{\calL}{{\cal L}}
\newcommand{\calM}{{\cal M}}
\newcommand{\calN}{{\cal N}}
\newcommand{\calO}{{\cal O}}
\newcommand{\calP}{{\cal P}}
\newcommand{\calQ}{{\cal Q}}
\newcommand{\calR}{{\cal R}}
\newcommand{\calS}{{\cal S}}
\newcommand{\calT}{{\cal T}}
\newcommand{\calU}{{\cal U}}
\newcommand{\calV}{{\cal V}}
\newcommand{\calW}{{\cal W}}
\newcommand{\calX}{{\cal X}}
\newcommand{\calY}{{\cal Y}}
\newcommand{\calZ}{{\cal Z}}

% bold letters (useful for random variables)
%----------------------------------------------
% \renewcommand{\a}{{\boldsymbol a}}
% \renewcommand{\b}{{\boldsymbol b}}
% \renewcommand{\c}{{\boldsymbol c}}
% \renewcommand{\d}{{\boldsymbol d}}
% \newcommand{\e}{{\boldsymbol e}}
% \newcommand{\f}{{\boldsymbol f}}
% \newcommand{\g}{{\boldsymbol g}}
% \newcommand{\h}{{\boldsymbol h}}
% \renewcommand{\i}{{\boldsymbol i}}
% \renewcommand{\j}{{\boldsymbol j}}
% \renewcommand{\k}{{\boldsymbol k}}
% \newcommand{\m}{{\boldsymbol m}}
% \newcommand{\n}{{\boldsymbol n}}
% \renewcommand{\o}{{\boldsymbol o}}
% \newcommand{\p}{{\boldsymbol p}}
% \newcommand{\q}{{\boldsymbol q}}
% \renewcommand{\r}{{\boldsymbol r}}
% \newcommand{\s}{{\boldsymbol s}}
% \renewcommand{\t}{{\boldsymbol t}}
% \renewcommand{\u}{{\boldsymbol u}}
% \renewcommand{\v}{{\boldsymbol v}}
% \newcommand{\w}{{\boldsymbol w}}
% \newcommand{\x}{{\boldsymbol x}}
% \newcommand{\y}{{\boldsymbol y}}
% \newcommand{\z}{{\boldsymbol z}}
% \newcommand{\A}{{\boldsymbol A}}
% \newcommand{\B}{{\boldsymbol B}}
% \newcommand{\C}{{\boldsymbol C}}
% \newcommand{\D}{{\boldsymbol D}}
% \newcommand{\E}{{\boldsymbol E}}
% \newcommand{\F}{{\boldsymbol F}}
% \newcommand{\G}{{\boldsymbol G}}
% \renewcommand{\H}{{\boldsymbol H}}
% \newcommand{\I}{{\boldsymbol I}}
% \newcommand{\J}{{\boldsymbol J}}
% \newcommand{\K}{{\boldsymbol K}}
% \renewcommand{\L}{{\boldsymbol L}}
% \newcommand{\M}{{\boldsymbol M}}
% \renewcommand{\O}{{\boldsymbol O}}
% \renewcommand{\P}{{\mathbb{P}}}
% \newcommand{\Q}{{\boldsymbol Q}}
% \newcommand{\R}{{\boldsymbol R}}
% \renewcommand{\S}{{\boldsymbol S}}
% \newcommand{\T}{{\boldsymbol T}}
% \newcommand{\U}{{\boldsymbol U}}
% \newcommand{\V}{{\boldsymbol V}}
% \newcommand{\W}{{\boldsymbol W}}
% \newcommand{\X}{{\boldsymbol X}}
% \newcommand{\Y}{{\boldsymbol Y}}
% \newcommand{\Z}{{\boldsymbol Z}}

% script letters
\newcommand{\scrA}{{\mathscr A}}
\newcommand{\scrB}{{\mathscr B}}
\newcommand{\scrC}{{\mathscr C}}
\newcommand{\scrD}{{\mathscr D}}
\newcommand{\scrE}{{\mathscr E}}
\newcommand{\scrF}{{\mathscr F}}
\newcommand{\scrG}{{\mathscr G}}
\newcommand{\scrH}{{\mathscr H}}
\newcommand{\scrI}{{\mathscr I}}
\newcommand{\scrJ}{{\mathscr J}}
\newcommand{\scrK}{{\mathscr K}}
\newcommand{\scrL}{{\mathscr L}}
\newcommand{\scrM}{{\mathscr M}}
\newcommand{\scrN}{{\mathscr N}}
\newcommand{\scrO}{{\mathscr O}}
\newcommand{\scrP}{{\mathscr P}}
\newcommand{\scrQ}{{\mathscr Q}}
\newcommand{\scrR}{{\mathscr R}}
\newcommand{\scrS}{{\mathscr S}}
\newcommand{\scrT}{{\mathscr T}}
\newcommand{\scrU}{{\mathscr U}}
\newcommand{\scrV}{{\mathscr V}}
\newcommand{\scrW}{{\mathscr W}}
\newcommand{\scrX}{{\mathscr X}}
\newcommand{\scrY}{{\mathscr Y}}
\newcommand{\scrZ}{{\mathscr Z}}

\newcommand{\im}{{\text{im }}}
\newcommand{\ip}[1]{\left\langle #1 \right\rangle}
\newcommand{\norm}[1]{\left\lVert #1 \right\rVert}
\newcommand{\abs}[1]{\left\lvert #1 \right\rvert}

\newcommand\blfootnote[1]{%
  \begingroup
  \renewcommand\thefootnote{}\footnote{#1}%
  \addtocounter{footnote}{-1}%
  \endgroup
}

\title{Functional Analysis Assignment 5}
\author{\textsc{Ashish Kujur}}

\date{}


\begin{document}
\maketitle
\section*{Note}
A checkmark $\checkmark$ indicates the question has been done.

\tableofcontents
\section{Question 1}
\horz
Let $V$ and $W$ be two NLS and $T: V \rightarrow W$ be a linear map. Show that $T$ is continuous if and only if $T$ maps every Cauchy sequence of $V$ to a Cauchy sequence of $W$.
\horz

\begin{proof}
    Let $V,W$ be two NLS and let $T: V\to W$ be a linear map.

    $\left( \Longrightarrow \right)$ Suppose that $T$ is continuous. Let $\left\{ x_{n} \right\}$ be a Cauchy sequence in $X$. We want to show that $\left\{ Tx_{n} \right\}$ is Cauchy sequence in $Y$. To do so, let $\varepsilon > 0$ be given. By the continuity of $T$, there is some $k> 0$ such that
    \begin{equation}
	\norm{Tx}\le k\norm{x} \text{ for every } x \in X\text{.}
	\label{eqn:q1-continuity-of-T}
    \end{equation}
    Since $\left\{ x_{n} \right\}$ is Cauchy, there is some $N\in \N$ such that
    \begin{equation}
	\norm{x_{n}-x_{m}} < \frac{\varepsilon}{k} \text{ for every } n,m \ge N
	\label{eqn:q1-cauchy}
    \end{equation}
    Thus, for every $n,m \ge N$, we have that
    \begin{align*}
	\norm{Tx_{n} - Tx_{m}} & \le k \norm{x_{n} - x_{m}} & \text{from } \ref{eqn:q1-continuity-of-T} \\
	& < \varepsilon & \text{from } \ref{eqn:q1-cauchy}
    \end{align*}
    This shows that $\left\{ Tx_{n} \right\}$ is Cauchy in $Y$.

    $\left( \Longleftarrow \right)$ We prove it by contrapostitively. Suppose that $T$ is not continuous. Then for every $k > 0$,
    \begin{align*}
	\norm{Tx} &> k \norm{x} \text{ for some } x \in X\text{.}
    \end{align*}
    Thus, for each $n\in \N$, we can find some $x_{n} \in X$ such that $\norm{Tx_{n}} > n^{2} \norm{x_{n}}$. Consider the sequence $\left\{ y_{n} \right\}$ in $V$ defined by 
    \begin{align*}
	y_{n} = \frac{x_{n}}{n\norm{x_{n}}} \text{ for each } n \in \N
    \end{align*}
    We now show that $\left\{ y_{n} \right\}$ is Cauchy. Let $\varepsilon > 0$ be given. Select $N \in \N$ such that $\frac{2}{N}<\varepsilon$. For $k \in \N$ and $n\ge N$, we have that
    \begin{align*}
    \norm{y_{n+k} - y_{m}} &= \norm{\frac{x_{n+k}}{(n+k)\norm{x_{n+k}}}-  \frac{x_{n}}{n\norm{x_{n}}}} \\
    &\le \frac{1}{n+k} + \frac{1}{n} \\
    &= \frac{2}{n} \le \frac{2}{N} < \varepsilon
    \end{align*}
    This shows that $\left\{ y_{n} \right\}$ is Cauchy but on the other hand, we have that
    \begin{align*}
	\norm{Ty_{n}} &= \norm{T\left( \frac{x_{n}}{n\norm{x_{n}}} \right)} >n
    \end{align*}
    This shows that $\left\{ Ty_{n} \right\}$ is unbounded, a property which Cauchy sequences cannot have.
\end{proof}

\section{Question 2}
\horz
Let $X$ and $Y$ be two Banach spaces and  $T: X \rightarrow Y$ be a continuous linear transformation. Show that there exist a constant $c>0$ such that $\|Tx\| \geqslant c \|x\|$ for all $x\in X$ if and only if $\ker T=\{0\}$ and $\im (T)$ is closed.

\horz

\begin{proof}[Solution]
    $\left( \Longrightarrow \right)$ Suppose that there is a constant $c>0$ such that $\norm{Tx} \ge c\norm{x}$ for all $x\in X$. 

    First, let us show that $\ker T = \left\{ 0 \right\}$. Let $x\in \ker T$. Then $Tx=0$. Then we have that $0=\norm{Tx}\ge c \norm{x}$ and hence $x=0$. 

    To show that that the image of $T$ is closed, let $\left( Tx_{n} \right)_{n\in \N}$ be a sequence converging to some $y \in Y$. We need to show that $y=Tx$ for some $x\in X$. 

    Since $\left( Tx_{n} \right)$ is convergent, it is Cauchy in $Y$. Therefore, we have that 
    \begin{align*}
	\norm{x_{n} -x_m} &\le \frac{1}{c} \norm{Tx_{n} - Tx_{m}}
    \end{align*}
    for all $m,n \in \N$. This shows that $\left( x_{n} \right)_{n\in \N}$ is Cauchy in $X$. Since $X$ is Banach, we have that $\left( x_{n} \right)_{n\in\N}$ converges to some $x\in X$. By continuity, we have that $\left( Tx_{n} \right)_{n\in \N}$ converges to $Tx$. By uniqueness of limits, we have that $Tx=y$.

    $\left( \Longleftarrow \right)$ If $X=\left\{ 0 \right\}$ then the result is trivial. Suppose that $X \ne \left\{ 0 \right\}$. Since $T: X \to Y$ is injective, we consider the map $T^{-1} : \im T \to X$. Note that $T$ is bounded linear transformation, thus, $T^{-1}$ is a bounded linear transformation by the inverse mapping theorem. (Quick remark: $\im T$ is Banach by virtue of being closed).

    Thus, we have that
    \begin{align*}
	\norm{x} &= \norm{T^{-1} \left( Tx \right)} \\
	&\le \norm{T^{-1}} \norm{Tx}
    \end{align*}
    for any $x\in X$.
    We will be done if we show that $\norm{T^{-1}} \ne 0$. Since $X$ is nonzero, $\im T$ is nonzero. Select a nonzero vector $y\in \im (T)$ such that $\norm{y} \le 1$. Thus, we have that $\norm{T^{-1}} \ge \norm{T^{-1} \left( y \right)}$. Hence $y=Tx$ for some nonzero $x\in X$. Thus, $\norm{T^{-1}} \ge \norm{x} >0$. This completes the proof.
\end{proof}

\section{Question 3}
\horz

Let $H$ be a Hilbert space and $T, S\in \mathcal B(H),$ satisfying $T\geqslant 0$ and $S\geqslant 0.$ Show that $T+S \geqslant 0.$

\horz

\begin{proof}
    Routine check. Follows immediately.
\end{proof}

\section{Question 4}
\horz
Let $\{ e_n:n\in\mathbb N\}$ be the standard Schauder basis for the Banach space $\ell^p(\mathbb N)$ where $1 \leqslant p < \infty.$  Show that $e_n \to 0$ in the weak topology of $\ell^p(\mathbb N)$ for every $p >1.$ But for $p=1,$ the sequence $e_n$ does not converges to $0$ in the weak topology of $\ell^1(\mathbb N).$
\horz
\begin{proof}
    First, we deal with the case when $1<p < +\infty$. It can be shown that
    \begin{equation*}
	(\ell ^{p} \left( N \right))^{*} = \left\{ L_{y} : y \in \ell ^{q} \left( \N \right) \right\}
    \end{equation*}
    where $L_{y}(x)=\sum_{i=1}^{\infty} x_{i}y_{i}, x\in \ell ^{q} \left( \N \right)$. Note that for each $y\in \ell ^{q} \left( \N \right)$ with $1 \le q < \infty$, we have that $y_{i} \to 0$ as $i \to \infty$. This is because $\sum_{i=1}^{\infty} \abs{y_{i}}^{q} < \infty$ for $y \in\ell^{q} \left( \N \right)$. 

    Now, let $y\in \ell ^{q} \left( \N \right)$. We have that 
    \begin{equation*}
	L_{y} \left( e_{n} \right) = y_{n} \to 0 \text{ as } n \to \infty.
    \end{equation*}
    This shows that $\left( e_{n} \right) $ converges to $0$ in the weak topology.

    Now, consider the case where $p=1$. Then we have
    \begin{equation*}
	\left( \ell ^{1} \left( \N \right) \right)^{*} = \left\{ L_{y} : y \in \ell ^{\infty} \left( \N \right) \right\}
    \end{equation*}
    where $L_{y}$ is as specified in the previous case. Let $y=\left( 1,1,1,\ldots \right)$. Then we have that
    \begin{equation*}
	L_{y} \left( e_{n} \right) = 1
    \end{equation*}
    for each $n \in \N$. Hence, we have that $(e_{n})$ does not converge to $0$ in the weak topology.
\end{proof}

\section{Question 5}
\horz
Let $T : \ell^2(\mathbb N) \to \ell^2(\mathbb N)$ be the linear map given by $T \Big((x_j)_{j\in\mathbb N}\Big) = \Big ( (\frac{x_j}{j})_{j\in \mathbb N}\Big).$ 

\begin{itemize}
\item[1.] Show that $T$ is continuous and injective.
\item[2.] Consider the map $T^{-1} : range (T) \to \ell^2(\mathbb N)$ given by $T^{-1}(Tf) =f$ for $f\in \ell^2(\mathbb N).$ Show that $T^{-1}$ is not continuous.
\item[3.] Conclude that $range (T)$ is not closed in $\ell^2(\mathbb N).$  
\end{itemize}
\horz

\begin{proof}[Solution]
    \begin{enumerate}
	\item This is clear from Holder's inequality.
	\item Let $k > 0$. Select $N \in \N$ such that $k < N$. Consider the sequence
	    \begin{align*}
		N=\norm{T^{-1}\left( \frac{e_{N}}{N} \right)} > k \norm{e_{N}}
	    \end{align*}
    This shows that $T$ is discontinuous.
\item If $\im T$ was closed then the Banach isomorphism theorem would tell us that $T^{-1} : \im T \to \ell ^{2} \left( \N \right)$ is continuous which would contradict item 2.
    
    \end{enumerate}
\end{proof}

\section{Question 6}
\horz
For each $y=(y_j)_{j\in\mathbb N}$ in $\ell^{\infty}(\mathbb N),$  consider the map $T_y : \ell^{1}(\mathbb N) \to \mathbb C$ defined by $$T_y(x) = \sum\limits_{j=1}^{\infty}x_jy_j,\,\,\,\, \,x = (x_j)_{j\in\mathbb N}\in \ell^{1}(\mathbb N).$$ Show that the map $y \to T_y$ is an isometry from $\ell^{\infty}(\mathbb N)$ onto  $\big(\ell^{1}(\mathbb N)\big)^*.$ Thus $\big(\ell^1(\mathbb N)\big)^*$ is isometrically isomorphic to $\ell^{\infty}(\mathbb N).$ 
\horz

\begin{proof}[Solution]
    Fix $y= \left( y_{j} \right)_{j\in \N} \in \ell ^{\infty} \left( \N \right)$. Consider the map 
    \begin{equation*}
	T_{y} \left( x \right) = \sum_{j=1}^{\infty} x_{j}y_{j}
    \end{equation*}
    for each $x= \left( x_{j} \right)_{j\in \N} \in \ell^{1} \left( \N \right)$.

    It is easy to see that this map is well defined, continuous linear functional by the Holder's inequality. Hence, we have that $T_{y} \in \left( \ell ^{1} \left( \N \right) \right)^{*}$. 

    Now, we show that the map $F: \ell ^{\infty} \left( \N \right) \to \left( \ell ^{1} \left( \N \right) \right)^{*}$ given by
    \begin{equation*}
	y \stackrel{F}{\longmapsto} T_{y}
    \end{equation*}
It is easy to see that the map is linear and all we need to show is that this map is an isometry and an isomorphism as well.
First, fix a $y \in \ell^{\infty} \left( \N \right)$ and observe that for any $x\in \ell ^{1} \left( \N \right)$ with $\norm{x}_{1} = 1$, we have that
\begin{align*}
    \abs{T_{y} (x)} &= \abs{\sum_{j=1}^{\infty} x_{j}y_{j}} &\\
    & \le \norm{x}_{1} \norm{y}_{\infty} & \text{Holder's inequality} \\
    &= \norm{y}_{\infty}
\end{align*}
Thus, taking supremum, we have from Question 4 that
\begin{equation*}
    \norm{T_{y}}_{(\ell ^{1} \left( \N \right))^{*}} \le \norm{y}_{\infty}
\end{equation*}
To show the reverse inequality, observe that for each $i \in \N$, we have that $\norm{e_{1}}_{1} = 1$ and hence, we have that
\begin{align*}
    \abs{T_{y} \left( e_{i} \right)} = \abs{y_{i}} \le \norm{T_{y}}_{(\ell ^{1} \left( \N \right))^{\infty}}
\end{align*}
for each $i\in \N$. Taking supremums over $i \in \N$, we have that
\begin{equation*}
    \norm{y}_{\infty} \le  \norm{T_{y}}_{(\ell ^{1} \left( \N \right))^{\infty}}
\end{equation*}
This shows that $y \mapsto T_{y}$ is an isometry. It remains to show that $F$ is an isomorphism. It suffices to show that $F$ is onto.

Let $T \in \left( \ell ^{1} \left( \N \right) \right)^{*}$. We need to find a $y \in \ell ^{\infty} \left( \N \right)$ such that $T=T_{y}$.

For each $i \in \N$, we define
\begin{equation*}
    y_{i} = T\left( e_{i} \right).
\end{equation*}

We now claim that $T=T_{y}$. It is easy to see that 
\begin{equation*}
    T\left( e_{i} \right) = T_{y} \left( e_{i} \right)
\end{equation*}
Note that $\operatorname{span} \left\{ e_{i} : i \in \N \right\} = c_{00}$ and since $\overline{c_{00}} = \ell ^{1} \left( \N \right)$, we have that $T=T_{y}$ as they agree on a dense subset.

This completes the proof of the claim.
\end{proof}

\section{Question 7}
\horz

Suppose $X$ is a finite dimensional normed linear space. Then show that the weak topology on $X$ and the norm topology on $X$ coincides.

\horz
\begin{proof}
    It is clear that the norm topology contains the norm topology. To show the reverse inclusion, we show that every open ball contains a basis element of the weak topology.

    Consider the open ball $B\left( 0,1 \right)$. Suppose that $X$ is of dimension $n$. Consider the linear functionals $f_{i} (x) = x_{i}$ for each $i=1, 2, \ldots, n$. Then it is easy to see that $\cap_{i=1}^{n} f_{i}^{-1} \left( B\left( 0, 1/2 \right) \right) \subset B(0,1)$. This completes the proof.
\end{proof}


\section{Question 8 \texorpdfstring{$\checkmark$}{\text{✓}}}
\horz
Let $X$ be a normed linear space and $F: X \to \mathbb C$ be a non zero linear functional. Suppose $F(x_0) \neq 0$ for some $x_0\in X.$ Show that $X= \ker F \oplus \mbox{span} \{x_0\},$ that is,
\begin{itemize}
\item[(i)] $\ker F \cap \mbox{span} \{x_0\} = \{0\}.$
\item[(ii)] $ X = \ker F + \mbox{span} \{x_0\}.$
\end{itemize}
Show that $F$ is continuous if and only if $\ker F$ is a closed subspace in $X.$ (Hint : Use the continuity of the projection map $\pi:X \to  X/\ker F$ defined by $\pi(x) = [x],\,x\in X.$ )
\horz
\begin{proof}
    Let $X$ be a normed linear space and $F: X \to \C$ be nonzero linear functional. Since $F$ is nonzero, there must be some $x_{0} \in X$ such that $F\left( x_{0} \right) \ne 0$. We now proceed to show that $X=\ker F \oplus \operatorname{span} \left\{ x_{0} \right\}$.

    We first show that $X=\ker F + \operatorname {span} \left\{ x_{0} \right\}$. Let $x\in X$. Then $F\left( x \right) \in \C$. Since $F\left( x_{0} \right) \ne 0$. There must be some $\lambda \in \C$ such that $F\left( x \right) = \lambda F \left( x_{0} \right)$. Thus, we have that $F\left( x-\lambda x_{0} \right) = 0$. Thus, $x-\lambda x_{0}  \in \ker F$. Hence, $x= \lambda x_{0} + y$ for some $y \in \ker F$. This shows that $X=\ker F + \operatorname{span} \left\{ x_{0} \right\}$. 

    Now, we proceed to show that $\ker F \cap \operatorname{span} \left\{ x_{0} \right\} = \left\{ 0 \right\}$. To do so, let $y\in \ker F \cap \operatorname{span } \left\{ x_{0} \right\}$. Then we have thath $y= \lambda x_{0}$ for some $\lambda \in \C$. Hence, we have that $F\left( y \right) = \lambda F\left( x_{0} \right) = 0$. Since $F\left( x_0 \right) \ne 0$, we have that $\lambda = 0$ and thus, $y=0$. This completes the proof of the claim.

    The above two paragraphs show that $X = \ker F \oplus \operatorname{span} \left\{ x_{0} \right\}$.

    Now, we proceed to show that $F$ is continuous iff $\ker F$ is a closed subspace of $X$. Let's begin the proof in the $\left( \Rightarrow \right)$ direction. Suppose that $F$ is continuous. Then we have that $\ker F = F^{-1} \left( \left\{ 0 \right\} \right)$ and hence it must be closed.
    
    To show the reverse direction, namely $\left( \Leftarrow \right)$, we first show that the projection map is continuous. First, we observe that for any $x\in X$, we have that
    \begin{align*}
	\norm{[x]} &= \inf_{y\in \ker F} \norm{x-y} & \text{by definition} \\
	&\le \norm{x} & 0 \in \ker F
    \end{align*}
    Now, this shows that the projection map $\pi : X \to X/\ker F$ is bounded and since it is a linear map, it is continuous.

    Now, consider the map $\tilde{T} : X / \ker F \to \C$ given by
    \begin{align*}
	[x] \stackrel{\tilde{T}}{\mapsto} F\left( x \right)
    \end{align*}
    We showed that $X= \ker F \oplus \operatorname{span} \left\{ x_{0} \right\}$. By the first isomorphism theorem for vector spaces, we have that $X/\ker F \cong \operatorname{span} \left\{ x_{0} \right\}$. This shows that $X/\ker F$ is finite dimensional. Since $\tilde{T}$ is linear and $X/\ker F$ is finite dimensional, we have thath $\tilde{T}$ is continuous.

    Observe that $T = \tilde{T} \circ \pi : X \to \C$ is continuous linear functional by virtue of being composition of two continuous linear maps. This completes the proof.
\end{proof}

\section{Question 9}

\horz

Let $H$ be a Hilbert space and $T\in \mathcal B(H).$ Suppose $X$ is an invertible operator in $\mathcal B(H).$ Then show that $\sigma (X^{-1}TX) = \sigma (T).$ ( In other words similar operators have same spectrum). 

\horz

\begin{proof}
    Let $X$ be an invertible operator in $\calB \left( H \right)$. Consider the following equivalence:
    \begin{align*}
	\alpha \not \in \sigma \left( T \right) &\Longleftrightarrow T-\alpha I \text{ is invertible} \\
	&\Longleftrightarrow X^{-1}T - \alpha X^{-1} \text{ is invertible} \\
	&\Longleftrightarrow X^{-1}TX - \alpha I \text{ is invertible} \\
	&\Longleftrightarrow \alpha \not \in \sigma \left( X^{-1}TX \right).
    \end{align*}
    This shows that $\sigma \left( X^{-1}TX \right) = \sigma \left( T \right)$.
\end{proof}

\section{Question 10 \texorpdfstring{$\checkmark$}{\text{✓}}}

\horz
Consider $C[0,1],$ the space of all complex valued continuous function on the interval $[0,1],$ equipped with the supremum norm, $\|\cdot\|_{\infty},$ that is, $\|f\|_{\infty}=\sup_{x\in [0,1]} |f(x)|.$ Let $S$ be the subset
    \begin{equation*}
	S=\left\{ f \in C[0,1] \, : \, \int_{0}^{1/2} f\left( x \right) dx - \int_{1/2}^{1} f \left( x \right) dx =1 \right\}
    \end{equation*}
    Show that the set $S$ is closed and convex but the distance is never achieved. That is, there is no $f\in S$ such that $\|f\|_{\infty} = d(0,S)$.
    
\horz
\begin{proof}[Solution]
   We begin by showing that $S$ is convex. Let $f, g \in S$ and $t\in [0,1]$. Then we have that 
    \begin{align*}
	\int_{0}^{1/2} \left( t f\left( x \right) + \left( 1-t \right) g\left( x \right)\right) dx - \int_{1/2}^{1} \left( t f\left( x \right) + \left( 1-t \right) g\left( x \right) dx  \right) &= t + (1-t) \\
	&= 1
    \end{align*}
    Note that the second equality follows by the virtue of $f,g \in S$.

    Now, we proceed to show that the $S$ is closed. Let $\left( f_{n} \right) $ be a sequence of functions in $S$ converging to $f \in C\left[ 0,1 \right]$. We need to prove that $f \in S$. Now convergence in supremum norm is the same as the uniform convergence, so, we have that following:
    \begin{align*}
	\lim_{n\to \infty}\left( \int_{0}^{1/2} f_{n}\left( x \right) dx - \int_{1/2}^{1} f_{n} \left( x \right) dx \right) =1
    \end{align*}
    implies 
    \begin{align*}
\int_{0}^{1/2} f\left( x \right) dx - \int_{1/2}^{1} f \left( x \right) dx =1 
\end{align*}
and thus $f \in S$.
Consider the zero function and the set $S$, we show that that there is no $f \in S$ such that $d (0,S) = d(f,0)=\norm{f}_{\infty}$. We show this in gentle steps as it follows.

Now, we proceed to show that $d\left( 0,S \right) = 1$. To do so, observe that we need to show that $\inf \left\{ \norm{f} _{\infty} : f \in S \right\} = 1$. First of all, if $f\in S$ then we have that 
\begin{align*}
    \int_{0}^{1/2} f - \int_{1/2}^{1} f =1 & \leadsto \abs {\int_{0}^{1/2} f - \int_{1/2}^{1} f } =1  \\
&\leadsto  \abs {\int_{0}^{1/2} f} +\abs{ \int_{1/2}^{1} f } \ge 1  \\
&\leadsto \norm{f}_{\infty} \ge 1
\end{align*}

Hence, we have that $1$ is a lowerbound for the set $S$. Now, let $\varepsilon > 0$ be given. We show that there is some function $f \in S$ such that $1 + \varepsilon > \norm{f}_{\infty}$. This will establish that $d\left( 0,S \right) = 1$. Select a $1/n < \varepsilon$.
Consider the function
\begin{align*}
    f(x) = \color{red}{tooLazyToFigureThisOut!}  
\end{align*}

It can be shown that this function $f$ has sup norm equals $1+\varepsilon$ and is a member of $S$.
Now, we proceed to show  that there is no function $f \in S$ such that $\norm{f}_{\infty} = 1$.

I sketch a idea of how to do this because it is way too long otherwise. Using the hypothesis show that 
\begin{equation*}
    \int_{0}^{1} f = 1/2
\end{equation*}
and 
\begin{equation*}
    \int_{1/2}^{1} f = -1/2
\end{equation*}
Further show that $f\equiv 1$ on $[0,1/2]$ and $f\equiv -1$ on $[1/2,1]$. But mother continuity won't let this happen.
\end{proof}

\section{Question 11}

\horz

Let $H$ be a Hilbert space and $T\in \mathcal B(H).$ Show that the following statements are equivalent :
\begin{itemize}
\item[(i)] $T$ is an isometry, that is, $T^*T= I.$
\item[(ii)] $\|Tx\|= \|x\|$ for every $x\in H.$
\item[(iii)] $\langle Tx,Ty\rangle = \langle x,y \rangle$ for every $x,y\in H.$
\end{itemize}

\horz

\begin{proof}
    \begin{description}
	\item[(i) $\Longrightarrow $ (ii):] Suppose that $T^{*}T=I$. Let $x\in H$. Then 
	    \begin{align*}
		\norm{Tx}^{2} &= \ip{Tx, Tx} \\
		&= \ip{T^{*}Tx, x} \\
		&= \ip{x,x} \\
		&= \norm{x}^{2}.
	    \end{align*}
	\item[(ii) $\Longrightarrow$ (iii)] Suppose that $\norm{Tx}=\norm{x}$ for each $x\in H$.
	    See Axler's Theorem 7.42.
    \item[(iii) $\Longrightarrow$ (i)] Suppose that $\ip{Tx, Ty}=\norm{x,y}$ for each $x,y\in H$. Let $x,y\in H$. Then we have
	\begin{align*}
	    \ip{(T^{*}T-I)x,y} &= \ip{T^{*}Tx-x,y} \\
	&= \ip{T^{*}Tx,y} - \ip{T^{*}Tx,y} \\ 
	&= \ip{Tx, Ty} - \ip{x,y} \\
	&= 0
	\end{align*}
	This shows that $T^{*}T=I$.
\end{description}
\end{proof}

\end{document}
