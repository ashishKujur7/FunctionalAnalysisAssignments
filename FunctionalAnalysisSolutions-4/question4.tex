\section{Question 4}
\horz

Let $(X,\|\cdot\|)$ be a Banach space with a Schauder basis, say $\{v_j: j\in\mathbb N\}.$ Thus for any $x\in X$, there exist  unique scalars $\{c_i(x): i\in\mathbb N\}$ such that $x= \sum_i c_i(x) v_i.$ Now consider the family of linear functional $P_i : X \to \mathbb F$ defined by $P_i(x) = c_i(x)$ for every $x\in X.$ Show that  $P_i$ is a continuous linear functional on $X$ for each $i\in \mathbb N.$

\horz

\begin{proof}
    By the uniqueness part of the Schauder basis, it is easy to see that each $P_{i} : X \to \F$ is indeed an linear functional. We are left to show that it is continuous. 

    We use the previous problem to complete this problem. Since $\norm{\cdot}$ and $\norm{\cdot}_{n}$ is equivalent, there exists constants $\alpha_1 , \alpha_2 > 0$ such that 
    \begin{equation*}
	\alpha_{1} \norm{x}_{n} \le \norm{x} \le \alpha_{2} \norm{x}_{n}
    \end{equation*}
    for each $x\in X$.

%    We first show that $P_{1} : X \to \F$ is continuous. Consider the following:
 %   \begin{align*}
%	\abs{c_{1} (x)} \norm{v_{1}} &= \norm{c_{1} \left( x \right) v_{1}} \\
%	&\le \norm{x}_{n} \\
%	&\le \frac{1}{\alpha_{1}} \norm{x} \\
%	\leadsto \abs{c_{1} \left( x \right)} &\le \frac{1}{\alpha_{1} \norm{v_{1}}} \norm{x}.
%    \end{align*}
%    This shows that $c_{1} (x)$ is continuous.

    We show that the map $\varphi_{k} : \left( V, \norm{\cdot} \right) \to \left( V, \norm{\cdot} \right)$ given by $\varphi _{k} \left( x \right) = \sum_{i=1}^{k} c_{i} \left( x \right) v_{i}$ is continuous. Let $x \in X$ and consider the following:
    \begin{align*}
	\norm{\varphi_{k} \left( x \right)} &= \norm{\sum_{i=1}^{k} c_{i} \left( x \right) v_{i}} \\
	&\le \norm{x}_{n} \\
	&\le \frac{1}{\alpha_{1}} \norm{x}.
    \end{align*}
    This shows that $\varphi_{k}$ is continuous for each $k\in \N$. Define $\varphi_{0} (x)=0$ for each $x\in X$. Notice that now, $P_{k} = \frac{1}{\norm{v_{k}}} \left( \varphi_{k} - \varphi_{k-1} \right)$ for each $k\in \N$. By continuity of norm and $\varphi_{k}$'s, we are done.
\end{proof}
