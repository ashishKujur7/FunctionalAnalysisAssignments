\section{Question 10}
\horz

Let $H$ be a Hilbert space and $T\in \mathcal B(H).$ Suppose $X$ is a unitary operator in $\mathcal B(H).$ Then show that 
\begin{itemize}
\item[(i)] $\|X^{-1}TX\| = \|T\|.$
\item[(ii)] $T$ is normal if and only if $X^{-1}TX$ is normal.
\item[(iii)] $T$ is self adjoint if and only if $X^{-1}TX$ is self adjoint.
\item[(iv)] $T \geqslant 0$ if and only if $X^{-1}TX \geqslant 0.$
\item[(v)] $T$ is an orthogonal projection if and only if $X^{-1}TX$ is orthogonal projection. 
\end{itemize}
\horz

\begin{proof}
    Let $H$ be a Hilbert space and $T \in \calB \left( H \right)$. Let $X$ be a unitary operator in $\calB \left( H \right)$. 

    \begin{enumerate}[label=(\roman*)]
	\item We intend to show that $\norm{X^{-1}TX}=\norm{T}$. Since $X$ is a unitary operator, we have that $X$ is an isometric isomorphism.
	    To this end, let $f\in H$ and consider the following:
	    \begin{align*}
		\norm{X^{-1}TXf}^{2} &= \norm{X^{*}TXf}^{2} \\
		&= \ip{X^{*}TXf, X^{*}TXf} \\
		&= \ip{TXf, XX^{*}TXf} \\
		&= \ip{TXf, TXf} \\
		&= \norm{TXf}^{2}.
	    \end{align*}
	    Thus, we have that
	    \begin{equation}
		\norm{X^{-1}TXf} = \norm{TXf}
		\label{eqn:10.1}
	    \end{equation}
	    for each $f\in H$.
	    Now, we claim that
	    \begin{equation}
		\left\{ \norm{TXf} : \norm{f}=1 \right\} = \left\{ \norm{Tf} : \norm{f}=1 \right\}.
		\label{eqn:10.2}
	    \end{equation}
	    To prove this:
	    \begin{description}
		\item[$\left( \subset \right):$]  Let $y=\norm{TXf}$ for some $f \in H$ with $\norm{f}=1$. Since $X$ is a unitary operator, we have that $\norm{Xf}=\norm{f}=1$, hence, $y$ is in the set of the right side of the equality of the aforementioned claim.
		\item[$\left( \supset \right):$] Let $y=\norm{Tg}$ for some $g\in H$ with $\norm{g}=1$. Since $X$ is unitary, it is invertible and hence $Xf=g$ for some $f\in H$. Since $X$ is unitary, we have that $\norm{f}=\norm{Xf}=\norm{g}=1$. Thus, $y$ is in the left side of the set in the aforementioned equality.
	    \end{description}
	    Thus, we have 
	    \begin{align*}
		\norm{X^{*}TX} &= \sup_{\norm{f}=1} \norm{X^{*}TXf} & \text{(by definition)}\\
		&= \sup_{\norm{f}=1} \norm{TXf} & \text{(see \ref{eqn:10.1})}\\
		&= \sup_{\norm{f}=1} \norm{Tf} & \text{(see \ref{eqn:10.2})} \\
		&= \norm{T}. & \text{(by definition)}
	    \end{align*}
	    This completes the proof.
	\item 
	    \begin{description}
		\item[$\left( \Longrightarrow \right)$] Assume that $T$ is normal. Consider the following:
		    \begin{align*}
			\left( X^{*}TX \right)\left( X^{*}TX \right)^{*} &=  X^{*}TXX^{*}T^{*}X \\
			&= X^{*}TT^{*}X
		    \end{align*}

		    and on the other hand, we have
		    \begin{align*}
			\left( X^{*}TX \right)^{*} \left( X^{*}TX \right) &=  X^{*}TXX^{*}TX \\
			&= X^{*}TT^{*}X \\
		    \end{align*}
		    Note that the two computations above show that $X^{*}TX$ is normal by assuming $T$ is normal.
		\item[$\left( \Longleftarrow \right)$] The computation in the other direction, normality of $X^{*}TX$ and the invertibility of $X$ imply that $T$ is normal.
	    \end{description}
	    This completes the proof.
	\item Skipping because \textsc{Easy!}.

	\item Skipping because \textsc{Easy!}.

	\item Skipping because \textsc{Easy!}.
    \end{enumerate}
\end{proof}
